\documentclass{beamer}

% Konfiguracja języka polskiego
\usepackage[T1]{fontenc}
\usepackage[utf8]{inputenc}
\usepackage[polish]{polski}

% Wybór motywu prezentacji
\usetheme{Madrid}
\usecolortheme{whale}

% Dane strony tytułowej
\title{System analizy telemetrii i strategii wyścigowej Formuły 1}
\author{Paweł Janduła, Mikołaj Mroczek, Szymon Dybał}
\date{\today}

\begin{document}

% Slajd tytułowy
\begin{frame}
    \titlepage
\end{frame}

% Slajd: Tech Stack
\begin{frame}{Stack technologiczny}
    \begin{itemize}
        \item \textbf{Język:} Python 3.10+
        \item \textbf{Dane:} OpenF1 API (darmowe, publiczne dane)
        \item \textbf{Biblioteki:}
            \begin{itemize}
                \item \texttt{requests} (obsługa API)
                \item \texttt{matplotlib} (generowanie wykresów)
                \item \texttt{numpy} / \texttt{pandas} (obliczenia matematyczne)
            \end{itemize}
    \end{itemize}
\end{frame}

% Slajd: Cel i Funkcjonalność
\begin{frame}{Cel i funkcjonalność projektu}
    \begin{itemize}
        \item \textbf{Cel:} Stworzenie narzędzia konsolowego do profesjonalnej analizy danych F1
        \item \textbf{Tryby pracy:}
            \begin{itemize}
                \item \textbf{Kwalifikacje:} Analiza „Ultimate Lap”, telemetria (punkty hamowania, otwarcie przepustnicy)
                \item \textbf{Wyścig:} Strategie oponiarskie, degradacja, regularność (Consistency Score)
            \end{itemize}
        \item \textbf{Interfejs:} Rankingi całej stawki oraz pojedynki Head-to-Head (np. Verstappen vs Hamilton)
    \end{itemize}
\end{frame}

% Slajd: Architektura Systemu
\begin{frame}{Architektura systemu}
    \begin{itemize}
        \item \textbf{Założenie:} Kod elastyczny i łatwo rozszerzalny (Open/Closed Principle)
        \item \textbf{Podział na warstwy:}
            \begin{itemize}
                \item \textbf{Warstwa kreacyjna:} Zarządzanie typami sesji
                \item \textbf{Warstwa procesowa:} Standaryzacja generowania raportów
                \item \textbf{Warstwa logiki:} Wymienne algorytmy analityczne
                \item \textbf{Warstwa danych:} Izolacja zewnętrznych API
            \end{itemize}
    \end{itemize}
\end{frame}

% Slajd: Analityka w praktyce
\begin{frame}{Analityka w praktyce}
    W ramach systemu realizowane są następujące analizy:
    \begin{enumerate}
        \item \textbf{Ultimate Lap:} Wyznaczanie teoretycznego najlepszego okrążenia
        \item \textbf{Consistency Score:} Wskaźnik regularności kierowcy
        \item \textbf{Tyre Degradation:} Analiza trendu zużycia opon
        \item \textbf{Wizualizacja strategii:} Wykresy generowane przy użyciu Matplotlib
    \end{enumerate}
\end{frame}

% Slajd: Wzorzec 1 - Abstract Factory
\begin{frame}{Wzorzec 1: Abstract Factory (Fabryka Abstrakcyjna)}
    \begin{itemize}
        \item \textbf{Elementy:} \texttt{ReportFactory}, \texttt{QualifyingFactory}, \texttt{RaceFactory}
        \item \textbf{Kontekst F1:} Weekend wyścigowy obejmuje różne typy sesji (Treningi, Kwalifikacje, Wyścig)
        \item \textbf{Zastosowanie:} Pozwala tworzyć raporty bez wiedzy o typie przetwarzanych danych sesji. \texttt{QualifyingFactory} konfiguruje system pod kątem telemetrii, a \texttt{RaceFactory} pod kątem degradacji opon
    \end{itemize}
\end{frame}

% Slajd: Wzorzec 2 - Template Method
\begin{frame}{Wzorzec 2: Template Method (Metoda Szablonowa)}
    \begin{itemize}
        \item \textbf{Elementy:} \texttt{RaceReportTemplate}
        \item \textbf{Kontekst F1:} Proces generowania raportu jest stały: Pobierz dane $\rightarrow$ Przelicz $\rightarrow$ Wyświetl
        \item \textbf{Uzasadnienie:} Klasa bazowa definiuje szkielet (\texttt{generate\_report()}), a klasy pochodne (np. \texttt{VisualChartReport}) nadpisują tylko sposób prezentacji danych
    \end{itemize}
\end{frame}

% Slajd: Wzorzec 3 - Strategy
\begin{frame}{Wzorzec 3: Strategy (Strategia)}
    \begin{itemize}
        \item \textbf{Elementy:} \texttt{AnalysisStrategy}, \texttt{TyreDegradationStrategy}
        \item \textbf{Kontekst F1:} Analityka opiera się na zmiennych algorytmach obliczania formy kierowcy
        \item \textbf{Uzasadnienie:} Pozwala na wymianę algorytmu analizy w locie (np. zmiana ze zużycia opon na analizę zużycia paliwa) bez modyfikacji struktury raportów (Zasada Open/Closed)
    \end{itemize}
\end{frame}

% Slajd: Wzorzec 4 - Composite
\begin{frame}{Wzorzec 4: Composite (Kompozyt)}
    \begin{itemize}
        \item \textbf{Elementy:} \texttt{TelemetryComposite} (Throttle, Brake, Gear)
        \item \textbf{Kontekst F1:} Telemetria składa się z wielu wymiarów analizowanych oddzielnie lub łącznie
        \item \textbf{Uzasadnienie:} Pozwala traktować pojedynczy algorytm (np. \texttt{BrakeAnalysis}) oraz grupę algorytmów (\texttt{TelemetryComposite}) jednakowo przez wspólną metodę \texttt{.calculate()}
    \end{itemize}
\end{frame}

% Slajd: Wzorzec 5 - Facade
\begin{frame}{Wzorzec 5: Facade (Fasada)}
    \begin{itemize}
        \item \textbf{Elementy:} \texttt{F1DataFacade}
        \item \textbf{Kontekst F1:} Biblioteki danych F1 (np. FastF1) wymagają złożonej synchronizacji wielu źródeł
        \item \textbf{Uzasadnienie:} Fasada ukrywa złożoność API zewnętrznych. Reszta systemu korzysta z prostej metody \texttt{get\_session\_data()}, co ułatwia podmianę źródła danych w przyszłości
    \end{itemize}
\end{frame}

% Slajd: Integracja wzorców
\begin{frame}{Integracja wzorców w systemie}
    Współpraca komponentów w pełnym cyklu analizy:
    \begin{enumerate}
        \item \textbf{Facade} -- pobieranie i synchronizacja surowych danych
        \item \textbf{Factory} -- tworzenie instancji odpowiedniego raportu dla sesji
        \item \textbf{Template Method} -- sterowanie etapami generowania raportu
        \item \textbf{Strategy / Composite} -- wykonanie specjalistycznych obliczeń
    \end{enumerate}
\end{frame}

\end{document}